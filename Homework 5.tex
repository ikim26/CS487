\documentclass[12pt]{article}
\usepackage [margin=1in]{geometry}
\usepackage{amsmath, amssymb}

\begin{document}

\noindent Isaac Kim

\noindent CS 487

\noindent Professor Baldimtsi

\noindent Apr 16, 2023

\begin{center} 
\textbf{Homework 5}
\end{center}

\noindent \textbf{1.} Let $G$ be $\{0,1\}^n$ under XOR operation.
\\
\\
\noindent Closure: Let $g_1, g_2 \in G$. Then, $|g_1| =  |g_2| = n$. For all $g_1, g_2 \in G$, $g_1 \oplus g_2 \in G$ since $|g_1 \oplus g_2| = n$.
\\
\\
\noindent Identity: Consider $e  = 0^n \in G$. Then, for all $g \in G$, $e \oplus g = g$.
\\
\\
\noindent Inverse: Let $g \in G$ and $g^{-1} = g$. Then, $g \oplus g^{-1} = 0^n = e$.
\\
\\
\noindent Associativity: Let $g_1, g_2, g_3 \in G$. Then, $(g_1 \oplus g_2) \oplus g_3 = g_1 \oplus (g_2 \oplus g_3)$.
\\
\\
\noindent Commutativity: Let $g_1, g_2 \in G$. Then, $g_1 \oplus g_2 = g_2 \oplus g_1$.

\noindent Hence, $G$ is an abelian group.
\\
\\
\noindent \textbf{2)} Using algorithm B.13 from the textbook, $3^{1500}$ mod 100 can be computed as follows:

\noindent Initial variable values: $a = 3, b = 1500, N = 100$
\\
\\
\noindent Loop 1: $x = 3, t = 1$, $b = 1500$ is not odd

$t=1, x = 3^2$ mod $100 = 9, b = 750$
\\
\\
\noindent Loop 2: $x = 9, t = 1$, $b = 750$ is not odd

$t=1, x = 9^2$ mod $100 = 81, b = 375$
\\
\\
\noindent Loop 3: $x = 81, t = 1$, $b = 375$ is odd

$t = 1 \cdot 81$ mod $100 = 81, x = 81^2$ mod $100 = 61, b = (375-1)/2 = 187$
\\
\\
\noindent Loop 4: $x = 61, t = 81$, $b = 187$ is odd

$t = 81 \cdot 61$ mod 100 = 41, $x = 61^2$ mod $100 = 21, b = (187-1)/2 = 93$
\\
\\
\noindent Loop 5: $x = 21, t = 41$, $b = 93$ is odd

$t = 41 \cdot 21$ mod 100 = 61, $x = 21^2$ mod $100 = 41, b = (93-1)/2 = 46$
\\
\\
\noindent Loop 6: $x = 41, t = 61$, $b = 46$ is not odd

$t = 61, x = 41^2$ mod $100 = 81, b = 46/2 = 23$
\\
\\
\noindent Loop 7: $x = 81, t = 61$, $b = 23$ is odd

$t = 61 \cdot 81$ mod 100 = 41, $x = 81^2$ mod $100 = 61, b = (23-1)/2 = 11$
\\
\\
\noindent Loop 8: $x = 61, t = 41$, $b = 11$ is odd

$t = 41 \cdot 61$ mod 100 = 1, $x = 61^2$ mod $100 = 21, b = (11-1)/2 = 5$
\\
\\
\noindent Loop 9: $x = 21, t = 1$, $b = 5$ is odd

$t = 1 \cdot 21$ mod 100 = 21, $x = 21^2$ mod $100 = 41, b = (5-1)/2 = 2$
\\
\\
\noindent Loop 10: $x = 41, t = 21$, $b = 2$ is not odd

$t = 21$, $x = 41^2$ mod $100 = 81, b = 2/2 = 1$
\\
\\
\noindent Loop 11: $x = 81, t = 21$, $b = 1$ is odd

$t = 21 \cdot 81$ mod 100 = 1, $x = 81^2$ mod $100 = 61, b = (1-1)/2 = 0$

\noindent Since b = 0, we return $t = 1$. Hence, $\mathbf{3^{1500}}$ \textbf{mod 100 = 1}.
\\
\\
\noindent \textbf{3)} $\mathbb{Z}_{15}$ = $\{0,1,2,3,4,5,6,7,8,9,10,11,12,13,14\}$, order = 15.
\\
\\
\noindent Let $\mathbb{Z}_{15}^{-1}$ be the set of inverses of the elements in $\mathbb{Z}_{15}$. Then, 
\begin{center}
$\mathbb{Z}_{15}^{-1}=\{1,0,14,13,12,11,10,9,8,7,6,5,4,3,2\}$
\end{center}
The $i^{th}$ element of $\mathbb{Z}_{15}^{-1}$ is the inverse of the $i^{th}$ element of $\mathbb{Z}_{15}$. (for all $z_i \in \mathbb{Z}_{15}$ and $z_i^{-1} \in \mathbb{Z}_{15}^{-1}, z_i + z_i^{-1}$ mod 15 = 1).
\\
\\
Yes, $\mathbb{Z}_{15}$ is cyclic.
\\
\\
\noindent \textbf{4)} $\mathbb{Z}_{15}^*$ = $\{1,2,4,7,8,11,13,14\}$, order = 8.
\\
\\
\noindent Let $\mathbb{Z}_{15}^{-1*}$ be the set of inverses of the elements in $\mathbb{Z}_{15}^*$. Then, 
\begin{center}
$\mathbb{Z}_{15}^{-1*}=\{1,8,4,13,2,11,7,14\}$
\end{center}
The $i^{th}$ element of $\mathbb{Z}_{15}^{-1*}$ is the inverse of the $i^{th}$ element of $\mathbb{Z}_{15}^*$. (for all $z_i \in \mathbb{Z}_{15}^*$ and $z_i^{-1} \in \mathbb{Z}_{15}^{-1*}, z_i \cdot z_i^{-1}$ mod 15 = 1).
\\
\\
No, $\mathbb{Z}_{15}^*$ is not cyclic.
\\
\\
\noindent \textbf{5)} 

\noindent \textbf{5.1)} Let $k_b = a$ be Bob's key and $k_a = w_3 \oplus t$ be Alice's key. Then, $k_a = w_3 \oplus t = w_2 \oplus b \oplus t = w_1 \oplus t \oplus b \oplus t = w_1 \oplus b = a \oplus b \oplus b = a = k_b$.
\\
\\
\noindent \textbf{5.2)} Let $A$ be our adversary/eavesdropper. Then, $A$ can see the transcript which consists of $w_1, w_2$ and $w_3$. Note that $A$ does not know what $a,b,t$ are. By Kerckhoff's principle, $A$ knows the process behind generating keys. Then, $A$ can simply compute $w_1 \oplus w_2 \oplus w_3 = (a \oplus b) \oplus (a \oplus b \oplus t) \oplus (a \oplus t) = a \oplus b \oplus a \oplus b \oplus t \oplus a \oplus t = a \oplus a \oplus a \oplus b \oplus b \oplus t \oplus t = a = k$. Since $A$ can compute the key just from the transcript, our protocol is NOT secure.
\\
\\
\noindent \textbf{6)} 

\noindent \textbf{6.1)} $h_a = g^x, h_b = g^y$ for some $x,y \in \mathbb{Z}$.

\noindent In class, we did a proof by reduction in which we reduced the Diffie-Hellman protocol to the DDH problem, which states that it is hard for an adversary to distinguish $DH_g(h_1,h_2) = g^{xy}$ from a uniform element of $g$, given $g, h_1=g^x, h_2=g^y, x,y \in \mathbb{Z}$.
\\
\\
\noindent Then, even if an eavesdropping adversary eavesdrops on the exchange of $g, h_a, h_b$, since this problem is hard, (hard to compute $DH_g(h_a,h_b) = g^{xy}$ given $g, h_a, h_b$), computing $k_A = (g^x)^y = (g^y)^x = k_B$ is also hard. Hence, an eavesdropping adversary cannot simply compute the key.
\\
\\
\noindent \textbf{6.2)} To compute $m = Dec(sk,c)$, compute $k = c_1^{sk} = (h_B)^x = (g^y)^x=g^{xy}$. Then, output $m = c_2 \oplus k$. This works because $c_2 = m \oplus k$ (from $Enc(pk,m)$) and so $c_2 \oplus k = m \oplus k \oplus k = m$.



\end{document}