\documentclass[12pt]{article}
\usepackage [margin=1in]{geometry}
\usepackage{amsmath, amssymb}

\begin{document}

\noindent Isaac Kim

\noindent CS 487

\noindent Professor Baldimtsi

\noindent Apr 29, 2023

\begin{center} 
\textbf{Homework 6}
\end{center}

\noindent \textbf{1.1} Let $x$ be the secret key, $c$ be our ciphertext and $c_1, c_2$ be the two halves of our ciphertext. Decryption works as follows: compute $c_1^x$ and compare with $c_2$. Then, there are two cases:
\\
Case 1: $c_2 = h^y$, $m$ = head

In this case, $c_1^x = (g^y)^x = g^{xy} = (g^x)^y = h^y = c_2$. Then, our message is "head".
\\
\\
Case 2: $c_2 = g^z$, $m$ = tail

In this case, $c_1^x = (g^y)^x = g^{xy} \neq g^z = c_2$. Then, our message is "tail". It is important to note that in this scenario, there is the possibility that $z = xy$ and we can decrypt incorrectly, but with a sufficiently large $q$, the probability of this happening is negligible.
\\
\\
\noindent \textbf{1.2} Contrapositive: If the above scheme is \emph{not} CPA secure, then DDH is "easy" in $G$. (DDH problem: Given $g, h_1, h_2$, distinguish DH$_g(h_1,h_2)$ from a uniform element of $G$)
\\
\\
This implies that there exists an adversary $A$ that breaks the CPA security of the scheme such that Pr[A wins] $\geq \frac{1}{2} + p(n)$ where $p(n)$ is non-negligible.
\\
\\
Let $B$ be our adversary for DDH. B will receive a DDH instance ($h_1, h_2, w$) as input from the challenger, where $h_1 = g^x, h_2 = g^y, w = g^{xy}$ or a random string. B sets $pk = h_1$ and sends $pk$ to $A$. $A$ sends its challenge $m_0 = head, m_1 = tail$. $B$ flips a bit $b \in \{0,1\}$ and computes encryption of $m_b$ such that $Enc_{pk}(m_b) = c*$ according to the encryption scheme in 1.1. Then, $B$ sends $c*$ to $A$. $A$ will output a bit $b'$. If $b' = b$, $B$ will output that $w$ is a "DDH tuple", else output that $w$ is random.
\\
\\
Analysis:

Case 1: $w = g^{xy}$, $pk = x$, $c_1 = g^y$, $c_2 = h^y = (g^x)^y = g^{xy}$

Here, Pr[B wins] = Pr[A wins]

Case 2: $w =$ random string

Here, Pr[A wins] = $\frac{1}{2}$ = Pr[B wins]
\\
\\
\noindent \textbf{2.1} Let $A$ be our adversary. Note, for some message $m$, $m' = 00000000||r||00000000||m$, $|m'| = ||N||$. $A$ gets access to some ciphertext $c = [m^e$ mod N]. $A$ then calculates $c_\beta = [\alpha^e \cdot c$ mod $N$] for some $\alpha \in \mathbb{Z}$. Then, $A$ sends $c_\beta$ to the decryption oracle. $Dec(c_\beta)$ will either return an error or give back a proper decyption $m_\beta' = [c_\beta^d$ mod N]. In the case that $A$ gets an error, it can just keep trying different $\alpha \in \mathbb{Z}$ until it gets back a valid $m_\beta'$. Then, since $m_\beta' = \alpha \cdot m'$, $A$ can compute $m_\beta'/\alpha$ and extract $m$ from $m'$.
\\
\\
\noindent \textbf{2.2} It is easier to construct a chosen-ciphertext attack on this scheme than on PKCS\#1 v1.5 because the padding is not of random length. In our scheme, the padding is of a set length, and $r$ is also short (8 random bits).
\\
\\
\noindent \textbf{3.1} Existential unforgability: An attacker should be unable to forge valid signature on \emph{any} message not signed by the sender.
\\
\\
Let $A$ be our attacker that is given the public key. $A$ can interact with the oracle Sign$_{sk}()$. By Kerckhoff's principle, $A$ knows how the signing works. Then, in order to forge message $M = m_1||m_2||...||m_n$, $A$ can interact with the sign oracle to sign message $M_\alpha = m_n||m_{n-1}||...||m_1$ and get back $\sum(M_\alpha) = \sigma(m_n),\sigma(m_{n-1}),...,\sigma(m_1)$. 
\\
\\Then, in order to forge $M$, $A$ can just output the reverse of $\sum(M_\alpha)$, or $\sum(M) = \sigma(m_1),\sigma(m_{2}),...,\sigma(m_n)$. $M$ was never queried to the sign oracle, and so $A$ has successfully forged $M$ and so this scheme does not satisfy existential unforgeability.
\\
\\
\noindent \textbf{3.2} Let $A$ be our attacker that is given the public key. $A$ can interact with the oracle Sign$_{sk}()$. By Kerckhoff's principle, $A$ knows how the signing works. Then, in order to forge message $M = m_1||m_2||...||m_n$, $A$ can interact with the sign oracle to sign messages $M_\alpha = m_1||m_2||...||m_{n-1}||0^k$, $M_\beta = 0^k||m_2||m_3||...||m_n$ and get back $\sum(M_\alpha) = \sigma(1||m_1),\sigma(2||m_2),...,\sigma(n||0^k)$ and $\sum(M_\beta) = \sigma(1||0^k),\sigma(2||m_2),\sigma(3||m_3),...,\sigma(n||m_n)$. 
\\
\\
Then, in order to forge $M$, $A$ can just replace the last $\sigma$ of $M_\alpha$ with the last $\sigma$ of $M_\beta$, namely replace $\sigma(n||0^k)$ with $\sigma(n||m_n)$. Then, we get $\sum(M) = \sigma(1||m_1),\sigma(2||m_2),...,\sigma(n||m_n)$ which is a valid signature for $M$, but $M$ was never queried to the sign oracle. Hence, $A$ has successfully forged $M$ and so this scheme does not satisfy existential unforgeability.
\\
\\
\noindent \textbf{4a)} Bob gets $m,\sigma =$Sign$_{sk}(m)=[f(m)^d$ mod $N]$ where $d$ is the private key. In order to verify that a message signature pair was indeed sent by Alice, Bob needs to compute $\sigma^e$ and checks if $\sigma^e = [f(m)$ mod $N]$ where $e$ is Alice's public key. Note that Bob also needs access to this new encoding function $f$.
\\
\\
\noindent \textbf{4b)} Fix PPT attacker $A,$ scheme $ \pi,$ message $m_0$

Define randomized experiment Forge$_{A,\pi,m_0}(n)$:

1. $pk,sk \leftarrow Gen(1^n)$

2. $A$ given $pk$, interacts with oracle Sign$_{sk}(\cdot)$; let $M$ be the set of messages sent to oracle, 

$m_0 \notin M$

3. $A$ outputs $(m_0, \sigma)$

4. $A$ succeeds and experiment evaluates to 1 if Vrfy$_{pk}(m_0, \sigma)=1$, $m_0 \notin M$

\noindent $\pi$ is target-message unforgeable if for all PPT attackers $A$, there is a negligible function $\epsilon$ such that:
\begin{center}
Pr[Forge$_{A,\pi,m_0}(n)=1] \leq \epsilon(n)$
\end{center}

\noindent \textbf{4c)} The existential unforgeability definition is stronger than target-message since it states that a scheme is unforgeable for \emph{any} message rather than a specific message. (existential unforgeability is target-message unforgeability for \emph{all} messages) 



\end{document}